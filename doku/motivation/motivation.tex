\chapter{Motivation}
\label{sec:motivation}

Sp�testens seitdem wir angefangen haben, t�glich mehrere Computer verschiedener
Gr��e vom Handy bis zur Full-Size-Workstation zu nutzen, haben wir uns bez�glich
des Austausches von Informationen und Dateien einige Probleme eingehandelt: Ohne
die Nutzung einer Cloud-L�sung ist es schwierig, den automatisierten Austausch
zwischen den Ger�ten zu realisieren. Das Ergebnis ist oft doppelte und
inkonsistente Datenhaltung auf unterschiedlichen Medien (USB-Stick, Handy,
Laptop etc.). Die grundlegenden Probleme, zu denen im vorliegenden Projekt
L�sungen erarbeitet werden sollen, lassen sich auf die folgenden drei
Stichpunkte reduzieren.

\begin{enumerate}
\item Konsistenz: Wie kann ein einheitlicher Versionsstand zwischen den Ger�ten
  hergestellt werden?
\item Verf�gbarkeit: Wie kann jedem Ger�t zu jedem Zeitpunkt die vollst�ndige
  Datenbasis zur Verf�gung gestellt werden?
\item Vertraulichkeit: Es ist nicht gew�nscht, pers�nliche Daten mit
  multinationalen Konzernen zu teilen. Wie kann dieses Ziel erreicht werden?
\end{enumerate}

Die L�sung f�r Problem 1 und 2 scheint wie bereits angedeutet einfach: Die
Nutzung einer Cloud-L�sung als verbindende Instanz zwischen allen
Ger�ten. Hierzu gibt es bereits fertige L�sungen wie z.B. Google Drive, iCloud,
Dropbox, OneDrive und viele mehr. All diese Fertigl�sungen haben jedoch den
Nachteil, dass der gesamte pers�nliche Datenbestand auf einen externen
(amerikanischen) Server ausgelagert wird und die dahinter stehenden Unternehmen
damit machen k�nnen, was sie m�chten. Ist dies f�r den Nutzer akzeptabel steht
der Nutzung einer solchen Cloud-L�sung nichts mehr im Wege. Ist man jedoch daran
interessiert, selbst �ber seine Daten zu verf�gen, muss etwas mehr Aufwand
getrieben werden.

Auch f�r Home-Clouds gibt es fertige L�sungen. Zu nennen ist hierbei die
Serverhardware von QNAP und Synology, welche eine Plug-and-Play-L�sung
darstellen. Gegen die Nutzung einer solchen Plattform sprechen jedoch folgende
Gr�nde:

\begin{itemize}
\item Preis-Leistungs-Verh�ltnis: Ein Synology Network-Attached-Storage-Server
  (NAS) mit einem Dual-Core Prozessor von Intel und 2GB RAM kostet ca. 450\euro{}. F�r
  den selben Preis ist es m�glich einen Server aus gebrauchten Teilen mit einem
  Hexa-Core-Xeon Prozessor und 64GB RAM zusammenzustellen.
\item Unflexibel: Die L�sungen von QNAP und Synology sind
  \textit{ausschlie�lich} als NAS konzipiert. Ein selbst
  aufgesetzter Server kann z.B. auch als Plattform f�r ein GitLab ein
  WordPress-Blog oder �hnliches dienen.
\item Begrenzte Erweiterbarkeit: Die Baugr��e eines typischen Plug-and-Play-NAS
  erlaubt keine gr��eren Upgrades bez�glich Prozessor, RAM und Massenspeicher.
\end{itemize}

Die Motivation, eine eigene Plattform aufzusetzen, ist damit begr�ndet. Im
folgenden Kapitel wird kurz die Hard- und Software-Architektur beschrieben.


%%% Local Variables:
%%% mode: latex
%%% TeX-master: "../doku_server"
%%% End:
