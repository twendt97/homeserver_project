\chapter{Einleitung}
\label{sec:einleitung}

Tr�umt nicht jeder an Informationstechnik interessierte Mensch von der
allumfassenden sicheren und unbegrenzten Cloud-L�sung f�r einen intuitiven
Workflow �ber alle Ger�te und Plattformen hinweg? Der Autor m�chte sich diesen
Traum erf�llen und seinen eigenen Webserver betreiben, um dort verschiedene
Dienste anbieten zu k�nnen. Aber Halt! Ist es nicht viel einfacher, diese L�sung
an extern zu vergeben und die Dienste von Apple, Google und wie sie nicht alle
hei�en zu nutzen? Einfacher in jedem Fall, doch m�chte man seine Daten nicht mit
einem internationalen Konzern teilen, so f�hrt kein Weg an der selbst gebauten
L�sung vorbei. Des Weiteren ist es mit den heute verfügbaren
Open-Source-Werkzeugen einfach geworden, einen eigenen Webserver zu betreiben
und diesen universell zu nutzen. Beispielsweise ist auch die Gestaltung eines
Webauftritts mit WordPress einfach realisierbar.

Im Folgenden wird zun�chst das Problem n�her erl�utert und die m�glichen bereits
vorhandenen L�sungen mit der geplanten selbst betriebenen L�sung
verglichen. Darauf aufbauend wird eine grobe Systemarchitektur skizziert, sowie
das Projekt in Arbeitspakete unterteilt.


%%% Local Variables:
%%% mode: latex
%%% TeX-master: "../doku_server"
%%% End:
